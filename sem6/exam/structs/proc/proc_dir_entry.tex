\strdef{proc\_dir\_entry}

\begin{lstlisting}
/*
 * This is not completely implemented yet. The idea is to
 * create an in-memory tree (like the actual /proc filesystem
 * tree) of these proc_dir_entries, so that we can dynamically
 * add new files to /proc.
 *
 * parent/subdir are used for the directory structure (every /proc file has a
 * parent, but "subdir" is empty for all non-directory entries).
 * subdir_node is used to build the rb tree "subdir" of the parent.
 */
struct proc_dir_entry {
	/*
	 * number of callers into module in progress;
	 * negative -> it's going away RSN
	 */
	atomic_t in_use;
	refcount_t refcnt;
	struct list_head pde_openers;	/* who did ->open, but not ->release */
	/* protects ->pde_openers and all struct pde_opener instances */
	spinlock_t pde_unload_lock;
	struct completion *pde_unload_completion;
	const struct inode_operations *proc_iops;
	union {
		const struct proc_ops *proc_ops;
		const struct file_operations *proc_dir_ops;
	};
	const struct dentry_operations *proc_dops;
	union {
		const struct seq_operations *seq_ops;
		int (*single_show)(struct seq_file *, void *);
	};
	proc_write_t write;
	void *data;
	unsigned int state_size;
	unsigned int low_ino;
	nlink_t nlink;
	kuid_t uid;
	kgid_t gid;
	loff_t size;
	struct proc_dir_entry *parent;
	struct rb_root subdir;
	struct rb_node subdir_node;
	char *name;
	umode_t mode;
	u8 flags;
	u8 namelen;
	char inline_name[];
} __randomize_layout;
\end{lstlisting}
