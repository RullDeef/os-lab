\strdef{workqueue\_struct}

\begin{lstlisting}
/*
 * The externally visible workqueue.  It relays the issued work items to
 * the appropriate worker_pool through its pool_workqueues.
 */
struct workqueue_struct {
	struct list_head	pwqs;		/* WR: all pwqs of this wq */
	struct list_head	list;		/* PR: list of all workqueues */

	struct mutex		mutex;		/* protects this wq */
	int			work_color;	/* WQ: current work color */
	int			flush_color;	/* WQ: current flush color */
	atomic_t		nr_pwqs_to_flush; /* flush in progress */
	struct wq_flusher	*first_flusher;	/* WQ: first flusher */
	struct list_head	flusher_queue;	/* WQ: flush waiters */
	struct list_head	flusher_overflow; /* WQ: flush overflow list */

	struct list_head	maydays;	/* MD: pwqs requesting rescue */
	struct worker		*rescuer;	/* MD: rescue worker */

	int			nr_drainers;	/* WQ: drain in progress */
	int			saved_max_active; /* WQ: saved pwq max_active */

	struct workqueue_attrs	*unbound_attrs;	/* PW: only for unbound wqs */
	struct pool_workqueue	*dfl_pwq;	/* PW: only for unbound wqs */

#ifdef CONFIG_SYSFS
	struct wq_device	*wq_dev;	/* I: for sysfs interface */
#endif
#ifdef CONFIG_LOCKDEP
	char			*lock_name;
	struct lock_class_key	key;
	struct lockdep_map	lockdep_map;
#endif
	char			name[WQ_NAME_LEN]; /* I: workqueue name */

	/*
	 * Destruction of workqueue_struct is RCU protected to allow walking
	 * the workqueues list without grabbing wq_pool_mutex.
	 * This is used to dump all workqueues from sysrq.
	 */
	struct rcu_head		rcu;

	/* hot fields used during command issue, aligned to cacheline */
	unsigned int		flags ____cacheline_aligned; /* WQ: WQ_* flags */
	struct pool_workqueue __percpu *cpu_pwqs; /* I: per-cpu pwqs */
	struct pool_workqueue __rcu *numa_pwq_tbl[]; /* PWR: unbound pwqs indexed by node */
};
\end{lstlisting}
