\strdef{worker}

\begin{lstlisting}
/*
 * The poor guys doing the actual heavy lifting.  All on-duty workers are
 * either serving the manager role, on idle list or on busy hash.  For
 * details on the locking annotation (L, I, X...), refer to workqueue.c.
 *
 * Only to be used in workqueue and async.
 */
struct worker {
	/* on idle list while idle, on busy hash table while busy */
	union {
		struct list_head	entry;	/* L: while idle */
		struct hlist_node	hentry;	/* L: while busy */
	};

	struct work_struct	*current_work;	/* L: work being processed */
	work_func_t		current_func;	/* L: current_work's fn */
	struct pool_workqueue	*current_pwq;	/* L: current_work's pwq */
	unsigned int		current_color;	/* L: current_work's color */
	struct list_head	scheduled;	/* L: scheduled works */

	/* 64 bytes boundary on 64bit, 32 on 32bit */

	struct task_struct	*task;		/* I: worker task */
	struct worker_pool	*pool;		/* A: the associated pool */
						/* L: for rescuers */
	struct list_head	node;		/* A: anchored at pool->workers */
						/* A: runs through worker->node */

	unsigned long		last_active;	/* L: last active timestamp */
	unsigned int		flags;		/* X: flags */
	int			id;		/* I: worker id */
	int			sleeping;	/* None */

	/*
	 * Opaque string set with work_set_desc().  Printed out with task
	 * dump for debugging - WARN, BUG, panic or sysrq.
	 */
	char			desc[WORKER_DESC_LEN];

	/* used only by rescuers to point to the target workqueue */
	struct workqueue_struct	*rescue_wq;	/* I: the workqueue to rescue */

	/* used by the scheduler to determine a worker's last known identity */
	work_func_t		last_func;
};
\end{lstlisting}
