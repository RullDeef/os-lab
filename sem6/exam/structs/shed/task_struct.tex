\strdef{task\_struct}

\begin{lstlisting}
struct task_struct {
	struct thread_info		thread_info;
	unsigned int			__state;

	void				*stack;
	refcount_t			usage;
	/* Per task flags (PF_*), defined further below: */
	unsigned int			flags;
	unsigned int			ptrace;

	/* smp fields skipped... */

	int				on_rq; /* run queue (per CPU) */

	int				prio;
	int				static_prio;
	int				normal_prio;
	unsigned int			rt_priority;

	/* ... */

	struct sched_info		sched_info;

	struct list_head		tasks;

	/* ... */

	struct mm_struct		*mm; /* memory mapping */
	struct mm_struct		*active_mm;

	int				exit_state;
	int				exit_code;
	int				exit_signal;

	/* The signal sent when the parent dies: */
	int				pdeath_signal;
	
	/* ... */

	pid_t				pid;
	pid_t				tgid;

	/* Real parent process: */
	struct task_struct __rcu	*real_parent;

	/* Recipient of SIGCHLD, wait4() reports: */
	struct task_struct __rcu	*parent;

	struct list_head		children;
	struct list_head		sibling;
	struct task_struct		*group_leader;

	/* PID/PID hash table linkage. */
	struct pid			*thread_pid;
	struct hlist_node		pid_links[PIDTYPE_MAX];
	struct list_head		thread_group;
	struct list_head		thread_node;

	/* PF_KTHREAD | PF_IO_WORKER */
	void				*worker_private;

	/*
	 * executable name, excluding path.
	 *
	 * - normally initialized setup_new_exec()
	 * - access it with [gs]et_task_comm()
	 * - lock it with task_lock()
	 */
	char				comm[TASK_COMM_LEN];

	struct nameidata		*nameidata;

#ifdef CONFIG_SYSVIPC /* System V IPC related stuff */
	struct sysv_sem			sysvsem;
	struct sysv_shm			sysvshm;
#endif

	/* Filesystem information: */
	struct fs_struct		*fs;

	/* Open file information: */
	struct files_struct		*files;

	/* Namespaces: */
	struct nsproxy			*nsproxy;

	/* Signal handlers: */
	struct signal_struct		*signal;
	struct sighand_struct __rcu		*sighand;
	sigset_t			blocked;
	sigset_t			real_blocked;
	/* Restored if set_restore_sigmask() was used: */
	sigset_t			saved_sigmask;
	struct sigpending		pending;

#ifdef CONFIG_AUDIT
#ifdef CONFIG_AUDITSYSCALL
	struct audit_context		*audit_context;
#endif
	kuid_t				loginuid;
	unsigned int			sessionid;
#endif

	/* Protection against (de-)allocation: mm, files, fs, tty, keyrings, mems_allowed, mempolicy: */
	spinlock_t			alloc_lock;

	/* ... */

	/*
	 * I/O subsystem state of the associated processes.  It is refcounted
	 * and kmalloc'ed. These could be shared between processes.
	 */
	struct io_context		*io_context;

	/* ... */

	/* CPU-specific state of this task: */
	struct thread_struct		thread;
};
\end{lstlisting}
