\strdef{vfsmount}

\begin{lstlisting}
#define MNT_NOSUID	    0x01
#define MNT_NODEV	    0x02
#define MNT_NOEXEC	    0x04
#define MNT_NOATIME	    0x08
#define MNT_NODIRATIME	0x10
#define MNT_RELATIME	0x20
#define MNT_READONLY	0x40	/* does the user want this to be r/o? */
#define MNT_NOSYMFOLLOW	0x80

#define MNT_SHRINKABLE	0x100
#define MNT_WRITE_HOLD	0x200

#define MNT_SHARED	    0x1000	/* if the vfsmount is a shared mount */
#define MNT_UNBINDABLE	0x2000	/* if the vfsmount is a unbindable mount */
/*
 * MNT_SHARED_MASK is the set of flags that should be cleared when a
 * mount becomes shared.  Currently, this is only the flag that says a
 * mount cannot be bind mounted, since this is how we create a mount
 * that shares events with another mount.  If you add a new MNT_*
 * flag, consider how it interacts with shared mounts.
 */
#define MNT_SHARED_MASK	        (MNT_UNBINDABLE)
#define MNT_USER_SETTABLE_MASK  (MNT_NOSUID | MNT_NODEV | MNT_NOEXEC \
                				 | MNT_NOATIME | MNT_NODIRATIME | MNT_RELATIME \
			                	 | MNT_READONLY | MNT_NOSYMFOLLOW)
#define MNT_ATIME_MASK          (MNT_NOATIME | MNT_NODIRATIME | MNT_RELATIME )

#define MNT_INTERNAL_FLAGS      (MNT_SHARED | MNT_WRITE_HOLD | MNT_INTERNAL | \
		                	    MNT_DOOMED | MNT_SYNC_UMOUNT | MNT_MARKED | \
			                    MNT_CURSOR)

#define MNT_INTERNAL	    0x4000

#define MNT_LOCK_ATIME		0x040000
#define MNT_LOCK_NOEXEC		0x080000
#define MNT_LOCK_NOSUID		0x100000
#define MNT_LOCK_NODEV		0x200000
#define MNT_LOCK_READONLY	0x400000
#define MNT_LOCKED		    0x800000
#define MNT_DOOMED		    0x1000000
#define MNT_SYNC_UMOUNT		0x2000000
#define MNT_MARKED		    0x4000000
#define MNT_UMOUNT		    0x8000000
#define MNT_CURSOR		    0x10000000

struct vfsmount {
	struct dentry *mnt_root;	/* root of the mounted tree */
	struct super_block *mnt_sb;	/* pointer to superblock */
	int mnt_flags;
	struct user_namespace *mnt_userns;
} __randomize_layout;
\end{lstlisting}